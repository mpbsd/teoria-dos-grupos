\section{Produtos semidiretos}\label{sec:produtos-semidiretos}

\begin{definition}
  Sejam $H,K$ grupos e
  \[
    H\to{Aut(K)},\quad{u\mapsto\widehat{u}},
  \]
  uma ação de $H$ à direita de $K$ por automorfismos deste último, que aqui denotaremos por
  \[
    \widehat{u}:K\to{K},\quad{v\mapsto{v^{u}}}.
  \]
  O produto semidireto $H\ltimes{K}$, de $K$ por $H$, é o produto cartesiano
  \[
    H\times{K}
    =
    \left\{
      (u,v)
      :
      u\in{H},v\in{K}
      \right\},
  \]
  munido da operação binária
  \[
    (x,y)(z,w)=(xz,y^{z}w),
  \]
  assim definida para todos $(x,y),(z,w)\in{H\times{K}}$.
\end{definition}

O produto semidireto $H\ltimes{K}$ dos grupos $H,K$ é, também, um grupo. Com efeito, dados $(r,s),(t,u)$ e $(v,w)$ quaisquer em $H\ltimes{K}$, tem-se que:
\begin{align*}
  (r,s)\cdot{(t,u)(v,w)}
  &=(r,s)(tv,u^{v}w)                 \\
  &=(r\cdot{tv},s^{tv}\cdot{u^{v}w}) \\
  &=(rt\cdot{v},(s^{t}u)^{v}w)       \\
  &=(rt,s^{t}u)(v,w)                 \\
  &=(r,s)(t,u)\cdot{(v,w)},
\end{align*}

\[
  (r,s)(1,1)=(r1,s^{1}1)=(r,s),
\]

\[
  (r,s)(r^{-1},s^{-r^{-1}})=(rr^{-1},s^{r^{-1}}s^{-r^{-1}})=(1,(ss^{-1})^{r^{-1}})=(1,1),
\]

Por último, observe que:

\[
  (1,s)^{(r,1)}=(r,1)^{-1}(1,s)(r,1)=(r^{-1},1)(r,s^{r})=(r^{-1}r,s^{r})=(1,s^{r}),
\]

\begin{proposition}
  Sejam $G$ um grupo e $M,N$ subgrupos de $G$, sendo:
  \[
    M\cdot{N}=G,
    \quad
    M\cap{N}=\left\{1\right\}
    \quad\text{e}\quad
    N\unlhd{G}.
  \]
  Então, $G\simeq{M\ltimes{N}}$.
\end{proposition}

\begin{proof}
  Neste caso, a função
  \[
    \phi:M\ltimes{N}\to{G},\quad{(m,n)\mapsto{mn}},
  \]
  é um isomorfismo de grupos. Ela é injetiva, pois $M\cap{N}=\left\{1\right\}$. É igualmente sobrejetiva, já que $M\cdot{N}=G$. Finalmente, tem-se que:
  \[
    (r,s)^{\phi}(u,v)^{\phi}=rs\cdot{uv}=ru\cdot{s^{u}v}=(ru,s^{u}v)^{\phi}=\left((r,s)(u,v)\right)^{\phi},
  \]
  para todos $(r,s),(u,v)\in{M\ltimes{N}}$.
\end{proof}

\begin{proposition}
  Seja $G=H\ltimes{K}$ o produto semidireto de um grupo $K$ por um grupo $H$. Se $\phi:H\to{L}$ e $\psi:K\to{L}$ são homomorfismos de grupos tais que
  \[
    \forall\,u\in{H}:\quad{\widehat{u}\cdot\psi=\psi\cdot\widehat{u^{\phi}}}\quad\text{(composição de funções)},
  \]
  onde
  \[
    \widehat{u^{\phi}}:L\to{L},\quad{x\mapsto{u^{-\phi}xu^{\phi}}},
  \]
  é a conjugação por $u^{\phi}$, então existe um homomorfismo de grupos $\varphi:G\to{L}$ tal que
  \[
    (u,1)^{\varphi}=u^{\phi}
    \quad\text{e}\quad
    (1,v)^{\varphi}=v^{\psi},
  \]
  para todos $u\in{H}$ e $v\in{K}$.
\end{proposition}

\begin{proof}
  A função
  \[
    \varphi:G\to{L},\quad{(u,v)\mapsto{u^{\phi}v^{\psi}}},
  \]
  está, evidentemente, bem definida. Tal definição foi pensada de modo que as seguintes identidades fossem cumpridas:
  \[
    \forall\,(u,v)\in{G}:
    \quad
    (u,1)^{\varphi}=u^{\phi}
    \quad\text{e}\quad
    (1,v)^{\varphi}=v^{\psi},
  \]
  Finalmente, observe que:
  \begin{align*}
    \left((r,s)(t,u)\right)^{\varphi}
    &=(rt,s^{t}u)^{\varphi}                                                                             \\
    &=(rt)^{\phi}(s^{t}u)^{\psi}                                                                        \\
    &=r^{\phi}t^{\phi}(s^{t})^{\psi}u^{\psi}                                                            \\
    &=r^{\phi}t^{\phi}(s^{\psi})^{t^{\phi}}u^{\psi}\quad(\text{pela hipótese sobre }\phi\text{ e }\psi) \\
    &=r^{\phi}s^{\psi}t^{\phi}u^{\psi}                                                                  \\
    &=(r,s)^{\varphi}(t,u)^{\varphi},
  \end{align*}
  para todos $(r,s),(t,u)\in{G}$.
\end{proof}
