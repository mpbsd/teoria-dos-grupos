\section{Teoremas de Sylow}\label{sec:teoremas-de-sylow}

\begin{definition}\label{def:p-grupo}
  Seja $p$ um número primo. Um $p$-grupo é um grupo $G$ em que todo elemento tem ordem igual a alguma potência de $p$.
\end{definition}

\begin{definition}
  Sejam $G$ um grupo e $p$ um número primo. Um subgrupo de $G$ que é também um $p$-grupo é dito um $p$-subgrupo de $G$.
\end{definition}

\begin{definition}\label{def:p-subgrupo-de-sylow}
  Sejam $G$ um grupo finito e $p$ um divisor primo da ordem de $G$. Um $p$-subgrupo de Sylow de $G$ é um $p$-subgrupo maximal $H$ de $G$, no seguinte sentido: se $K$ é um $p$-subgrupo de $G$ e $H\leqslant{K}$, então $H=K$ ou $K=G$.
\end{definition}

\begin{theorem}[Sylow]\label{thm:primeiro-teorema-de-sylow}
  Sejam $G$ um grupo e $p$ um divisor primo da ordem de $G$. Se $p^{\alpha}$ é a maior potência de $p$ que divide a ordem de $G$, isto é, se:
  \[
    \order{G}=p^{\alpha}m,\quad\text{onde}\quad(p,m)=1,
  \]
  então existe ao menos um subgrupo $H$ de $G$ de ordem $p^{\alpha}$.
\end{theorem}

\begin{proof}
  Seja
  \[
    \Omega(G,p)
    =
    \left\{
      X\subset{G}
      :
      \order{X}=p^{\alpha}
    \right\},
  \]
  a coleção dos subconjuntos de $G$ que possuem cardinalidade $p^{\alpha}$. Como se sabe, o número de elementos de $\Omega(G,p)$ é dado por:
  \[
    \order{\Omega(G,p)}
    =
    \binom{\order{G}}{p^{\alpha}}
    =
    \frac{\order{G}!}{p^{\alpha}!(\order{G}-p^{\alpha})!}
    =
    \frac{\order{G}}{p^{\alpha}}\frac{(\order{G}-1)}{p^{\alpha}-1}\cdots\frac{(\order{G}-p^{\alpha}+1)}{1}.
  \]
  Note que
  \[
    \forall\,i\in\{1,\ldots,\alpha\}:\quad
    \frac{\order{G}-i}{p^{\alpha}-i}\not\equiv{0}\pmod{p},
  \]
  Com efeito, se alguma potência inteira do primo $p$, digamos, $p^{\beta}$, divide a diferença $\order{G}-i$, então existe um número inteiro $q$ com que se possa escrever $p^{\alpha}m-i=p^{\beta}q$. Neste caso, tem-se que $(p^{\alpha-\beta}m-q)p^{\beta}=i$ e, ainda, como $p^{\alpha-\beta}m-q$ é um número inteiro, que $p^{\beta}$ divide $i$. Observe, finalmente, que $p^{\beta}$ precisa dividir $p^{\alpha}-i$ porque divide ambos $p^{\alpha}$ e $i$. Raciocinando de maneira análoga pode-se provar que, se $p^{\beta}$ divide $p^{\alpha}-i$, então ele também deve dividir $\order{G}-i$. Desta forma, $\frac{\order{G}-i}{p^{\alpha}-i}$ não é divisível por $p$.

  Considere a ação de $G$ à direita de $\Omega(G,p)$, dada por:
  \[
    \Omega(G,p)\times{G}\to\Omega(G,p),\quad{(X,g)\mapsto{X\cdot{g}}},
  \]
  onde $X\cdot{g}=\{xg:x\in{X}\}$. Pelas Proposições~\ref{prop:equivalence-relation-associated-with-an-action} e~\ref{prop:counting-orbits}, tem-se que:
  \[
    \order{\Omega(G,p)}=\sum_{i}\,[G:H_{i}],
  \]
  onde $H_{i}=\{g\in{G}:X_{i}\cdot{g}=X_{i}\}$ é o estabilizador de $X_{i}$ em $G$. Vê-se, assim, que existe um $X_{i_{0}}$ em $\Omega(G,p)$ para o qual $[G:H_{i_{0}}]\not\equiv{0}\pmod{p}$, pois $p$ não divide $\order{\Omega(G,p)}$. Faça $H=H_{i_{0}}$. Como $p$ é primo, divide o produto $\order{G}=[G:H]\order{H}$ e não divide $[G:H]$, ele precisa dividir $\order{H}$. Logo, $p^{\alpha}$ divide $\order{H}$ e, deste modo, $p^{\alpha}\leqslant\order{H}$. Finalmente, tome um elemento $a\in{X_{i_{0}}}$ qualquer e com ele defina a seguinte função:
  \[
    H\to{X_{i_{0}}},\quad{x\mapsto{ax}}.
  \]
  Tal função está bem definida pois $H$ é o estabilizador de $X_{i_{0}}$ em $G$, de sorte que $ax\in{X_{i_{0}}}\cdot{x}=X_{i_{0}}$ para todo $x\in{H}$. Como esta função é injetora, tem-se que $\order{H}\leqslant{X_{i_{0}}}=p^{\alpha}$. De $p^{\alpha}\leqslant\order{H}$ e $\order{H}\leqslant{p^{\alpha}}$ segue que $\order{H}=p^{\alpha}$. Isto conclui a presente demonstração.
\end{proof}

\begin{definition}
  Sejam $G$ um grupo finito e $p$ um divisor primo da ordem de $G$. Denotamos por $\text{Syl}_{p}(G)$ ao conjunto de todos os $p$-subgrupos de Sylow de $G$, e por $n_{p}$ ao número de $p$-subgrupos de Sylow de $G$, isto é, $n_{p}=\order{\text{Syl}_{p}(G)}$.
\end{definition}

\begin{proposition}\label{prop:normalizadores-de-p-subgrupos-de-sylow}
  Sejam $G$ um grupo finito e $p$ um divisor primo da ordem de $G$. Sejam $H$ um $p$-subgrupo de Sylow de $G$ e $N=N_{G}(H)$ o normalizador de $H$ em $G$. Então, $N_{G}(N)=N$, ou seja, $N$ é o seu próprio normalizador em $G$.
\end{proposition}

\begin{proof}
  Note que $H\leqslant{N}$ e, portanto, que $p$ divide a ordem de $N$. Se $x$ é um $p$-elemento de $N$ (isto significa que $\langle{x}\rangle$ é $p$-subgrupo de $N$), então $\langle{x}\rangle{H}=H\langle{x}\rangle$ e, assim, $\langle{x}\rangle{H}\leqslant{G}$. Suponha, por contradição, que $x\notin{H}$. Neste caso, tem-se que:
  \[
    \order{\langle{x}\rangle{H}}=\frac{\order{x}\order{H}}{\order{\langle{x}\rangle\cap{H}}}>\order{H}\quad\text{pois}\quad\langle{x}\rangle>\langle{x}\rangle\cap{H},
  \]
  ou seja, que $\langle{x}\rangle{H}$ é um $p$-subgrupo de $G$ cuja ordem é superior à de um $p$-subgrupo de Sylow de $G$. Esta contradição nos obriga a admitir que $x\in{H}$. Esta discussão nos permite concluir que:
  \[
    \forall\,y\in{N_{G}(N)}:\quad{H^{y}\leqslant{N^{y}=N}\implies{H^{y}=H}\implies{y\in{N_{G}(H)=N}}},
  \]
  de onde segue que $N_{G}(N)=N$.
\end{proof}

\begin{theorem}[Sylow]\label{thm:segundo-teorema-de-sylow}
  Sejam $G$ um grupo finito e $p$ um divisor primo da ordem de $G$. Então, todos os $p$-subgrupos de Sylow de $G$ são conjugados entre si.
\end{theorem}

\begin{proof}
  Sejam $H,K\in\text{Syl}_{p}(G)$ dois $p$-subgrupos de Sylow quaisquer de $G$. Seja $H^{G}=\{H^{g}:g\in{G}\}$ a $G$-órbita da ação por conjugação de $G$ à direita de $\text{Syl}_{p}(G)$. Observe que $p$ não divide $\order{H^{G}}$, pois:
  \[
    \order{H^{G}}=[G:N_{G}(H)]=\frac{[G:H]}{[N_{G}(H):H]}.
  \]
  Finalmente, considere a ação por conjugação de $K$ à direita de $H^{G}$:
  \[
    H^{G}\times{K}\to{H^{G}},\quad{(H^{x},y)\mapsto{H^{xy}}}.
  \]
  Da identidade
  \[
    \order{H^{G}}=\sum_{i}\,[K:K\cap{N_{G}(H^{x_{i}})}],
  \]
  segue que $p$ não divide $[K:K\cap{N_{G}(H^{x_{i_{0}}})}]$, para algum $i_{0}$. Como este número é uma potência de $p$, isto só pode acontecer se $[K:K\cap{N_{G}(H^{x_{i_{0}}})}]$ for igual a $1$, ou seja, se $K\leqslant{N_{G}(H^{x_{i_{0}}})}$. Portanto, $K\leqslant{H^{x_{i_{0}}}}$, como segue da Proposição~\ref{prop:normalizadores-de-p-subgrupos-de-sylow}. Como $\order{K}=\order{H^{x_{i_{0}}}}$, tem-se que $K=H^{x_{i_{0}}}$. Isto encerra esta demonstração.
\end{proof}

\begin{theorem}[Sylow]\label{thm:terceiro-teorema-de-sylow}
  Sejam $G$ um grupo finito e $p$ um divisor primo da ordem de $G$. Digamos que $\order{G}=p^{\alpha}m$, onde $\alpha,m\in{\mathbb{Z}}$, $\alpha>0$ e $(p,m)=1$. Então, $n_{p}\equiv{1}\pmod{p}$ e, além disso, $n_{p}$ divide $m$.
\end{theorem}
